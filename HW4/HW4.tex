\documentclass[12pt]{article}

\topmargin -40pt
\marginparwidth 0pt
\oddsidemargin  -40pt
\evensidemargin 0pt
\marginparsep 0pt
\linespread{1.5}
\textwidth 7.2 in
 \textheight  10 in
 \hoffset  0.1in

\usepackage{amsthm,amsmath,amssymb,amscd,verbatim,epsfig}
\usepackage{mathptmx}
\usepackage{amsfonts}
%\usepackage{setapace}
\usepackage{graphicx}
\usepackage{bm}
%\usepackage{CJK}
\usepackage{ulem}
\usepackage{multicol}
\usepackage{enumerate}
\usepackage{float}
\usepackage{fontspec}
\usepackage{xeCJK}
\setmainfont{Times New Roman}
\setCJKmainfont{TaipeiSansTCBeta-Regular}
\XeTeXlinebreaklocale "zh"
\XeTeXlinebreakskip = 0pt plus 1pt

\title{Homework 4 of Introduction to Analysis (I), Honor Class}
\author{AM15 黃琦翔 111652028}

\begin{document}
\maketitle
\begin{enumerate}
    \item Let $B = A \cap \mathbb{Q}^n$.
    Then, for any $p \in B$, we can find a $\epsilon\in \mathbb{Q} > 0$ s.t. $D(p, \epsilon) \subseteq B$.
    Since $\mathbb{Q}$ is countable, the number of open balls contains any points in $B$ is isomorphic to $\mathbb{Q}^n \oplus \mathbb{Q}$, which is also countable.
    Thus, we want to proof the union of the collection of open balls contains any points in $A$.

    For any $x \in A$, since $A$ is open, exists $\epsilon > 0$ s.t. $D(x, \epsilon) \subseteq A$. Taking $p \in D(x, \epsilon) \cap \mathbb{Q}^n$ with $d(p, x) < \dfrac{\epsilon}{2}$.
    Then, $D(p, \dfrac{\epsilon}{2})$ contains $x\implies$ every points in $A$ is contained in a open ball in $B$.
    Therefore, any $A \subseteq \mathbb{R}^n$, $A$ is union of a countable collection of open balls.

    \item\begin{enumerate}
        \item\begin{enumerate}
            \item $x$ is an accumulation point, for all $\epsilon > 0$, $D(x, \epsilon)/\lbrace x \rbrace \cap A \neq \emptyset\implies D(x, \epsilon) \cap A \neq \emptyset$.
            Thus, $x$ is a limit point.
            \item Let $M = \mathbb{R}, d(x, y) = |x-y|,\ A = \lbrace n \mid n \in \mathbb{N}\rbrace$, then $1$ is a limit point of $A$ since $B(0, \epsilon) \cap A \neq \emptyset$ for all $\epsilon > 0$.
            
            But for $0 < \epsilon < 1$, $B(1, \epsilon)/\lbrace 1 \rbrace \cap A = \emptyset$. Thus, limit point is not accumulation all the time.
        \end{enumerate}
        \item Assume there exists a limit point $x$ of $A$ is not in $A$.
        Then, $x \in A^c$. We want to show that $A$ is not close.
        \begin{quote}
            For $A^c$, since all neighborhood $U$ of $x$ has intersection with $A$, then $U \nsubseteq A^c$. 
            Thus, $A^c$ is not open.
        \end{quote}

        Therefore, if $A$ is close, $A$ contains all its limit points.
    \end{enumerate}
    \newpage

    \item \begin{enumerate}
        \item Since $x \in A_1'$, for all $\epsilon > 0$, $(D(x, \epsilon)/\lbrace x \rbrace) \cap A_1 \neq \emptyset$.
        Then, let $y \in (D(x, \epsilon)/\lbrace x \rbrace) \cap A_1$, $d(x, y) < \epsilon$ for all $\epsilon > 0\implies \inf\lbrace d(x, y) \mid y \in A_1\rbrace = 0$.

        \item \begin{enumerate}
            \item[($\subseteq$)] For $x \in B_n'$, for any $\epsilon > 0$, $(D(x, \epsilon) / \lbrace x \rbrace) \cap B_n = (D(x, \epsilon) / \lbrace x \rbrace) \cap (\cup A_i) \\
            = \cup((D(x, \epsilon) / \lbrace x \rbrace) \cap A_i) \neq \emptyset$.
            Thus, $(D(x, \epsilon) / \lbrace x \rbrace) \cap A_i\neq \emptyset$ for some $i\implies x \in \cup A_i'$.
            
            \item[($\supseteq$)] If $x \in A_i'$, for any $\epsilon > 0$, $(D(x, \epsilon)/\lbrace x \rbrace) \cap A_i \neq \emptyset\implies \emptyset \neq (D(x, \epsilon) / \lbrace x \rbrace) \cap (\cup A_i) \\
            = (D(x, \epsilon) / \lbrace x \rbrace) \cap B_n$.
        \end{enumerate}

        Thus, $B_n' = \cup A_i'$.
    \end{enumerate}

    \item False.
    
    Let $A_i = (1/i, 1)$. Then, $B = (0, 1)$. And for $0$, $0 \in B'$ but not in $\cup A_i'$.
\end{enumerate}
\end{document}
