\documentclass[12pt]{article}

\topmargin -40pt
\marginparwidth 0pt
\oddsidemargin  -40pt
\evensidemargin 0pt
\marginparsep 0pt
\linespread{1.5}
\textwidth 7.2 in
 \textheight  10 in
 \hoffset  0.1in

\usepackage{amsthm,amsmath,amssymb,amscd,verbatim,epsfig}
\usepackage{mathptmx}
\usepackage{amsfonts}
%\usepackage{setapace}
\usepackage{graphicx}
\usepackage{bm}
%\usepackage{CJK}
\usepackage{ulem}
\usepackage{multicol}
\usepackage{enumerate}
\usepackage{float}
\usepackage{fontspec}
\usepackage{xeCJK}
\setmainfont{Times New Roman}
\setCJKmainfont{TaipeiSansTCBeta-Regular}
\XeTeXlinebreaklocale "zh"
\XeTeXlinebreakskip = 0pt plus 1pt

\title{Homework 4 of Introduction to Analysis (I), Honor Class}
\author{AM15 黃琦翔 111652028}

\begin{document}
\maketitle
\begin{enumerate}
    \item Since $A \subseteq \mathbb{R}^n$ is open,
    then for any point $x \in A$, exists some $\epsilon > 0$ s.t. $B(x, \epsilon) \subset A$.

    \item\begin{enumerate}
        \item\begin{enumerate}
            \item $x$ is an accumulation point, for all $\epsilon > 0$, $D(x, \epsilon)/\lbrace x \rbrace \cap A \neq \emptyset\implies D(x, \epsilon) \cap A \neq \emptyset$.
            Thus, $x$ is a limit point.
            \item Let $M = \mathbb{R}, d(x, y) = |x-y|,\ A = \lbrace n \mid n \in \mathbb{N}\rbrace$, then $1$ is a limit point of $A$ since $B(0, \epsilon) \cap A \neq \emptyset$ for all $\epsilon > 0$.
            
            But for $0 < \epsilon < 1$, $B(1, \epsilon)/\lbrace 1 \rbrace \cap A = \emptyset$. Thus, limit point is not accumulation all the time.
        \end{enumerate}
        \item Assume there exists a limit point $x$ of $A$ is not in $A$.
        Then, $x \in A^c$. We want to show that $A$ is not close.
        \begin{quote}
            For $A^c$, since all neighborhood $U$ of $x$ has intersection with $A$, then $U \nsubseteq A^c$. 
            Thus, $A^c$ is not open.
        \end{quote}

        Therefore, if $A$ is close, $A$ contains all its limit points.
    \end{enumerate}
    \newpage

    \item \begin{enumerate}
        \item For $x \in \bar{A}_1 = A_1\cup A_1'$, then we only need to show that the accumulation point $x$ of $A_1\implies \inf\lbrace d(x, y)\mid y \in A_1\rbrace = 0$.
        \begin{quote}
            Assume $\inf\lbrace d(x, y)\mid y \in A_1\rbrace = \alpha > 0$, then $B(x, \epsilon) \cap A_1 = \emptyset$ for $0 < \epsilon < \alpha$.
            Implies that $x$ is not an accumulation point of $A_1$.
        \end{quote}
        Thus, if $x \in \bar{A}_1$, $\inf \lbrace d(x, y) \mid y \in A_1\rbrace = 0$. 

        \item \begin{enumerate}
            \item[($\subseteq$)] For $x \in B_n$, $D(x, \epsilon) \cap B_n \neq \emptyset$ for all $\epsilon > 0$.

            $D(x, \epsilon) \cap (\cup^{n}_{i=1} A_i) = \cup^{n}_{i=1} (D(x, \epsilon) \cap A_i) \neq \emptyset\implies D(x, \epsilon) \cap A_i\neq \emptyset$ for some $i \in [1, n]$.
            Thus, $x \in \bar{A}_i$ for any $\epsilon > 0$ and some $i \in [1, n] \implies x \in \cup_{i=1}^n \bar{A}_i$.
 
            \item[($\supseteq$)] For $x \in \cup_{i=1}^n \bar{A}_i$ and $\epsilon > 0$, $D(x, \epsilon) cap \bar{A}_i \neq \emptyset$ for some $i$.
            This implies that $D(x, \epsilon) \cap (\cup_{i=1}^n A_i) = D(x, \epsilon) \cap B_n \neq \emptyset\implies x \in \bar{B}_n$
        \end{enumerate}

        \item Using the same way in (b), we can know that $U_{i=1}^{\infty} A_i \subseteq B$.
        But the union of close sets may not be close for some time, then the inclusion can be proper. Like $A_i = (\dfrac{1}{I}, 1)$, $\bar{B} = [0, 1]$ and $\cup A_i = (0, 1]$.
    \end{enumerate}
\end{enumerate}
\end{document}
