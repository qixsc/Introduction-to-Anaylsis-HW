\documentclass[12pt]{article}

\topmargin -40pt
\marginparwidth 0pt
\oddsidemargin  -40pt
\evensidemargin 0pt
\marginparsep 0pt
\linespread{1.9}
\textwidth 7.2 in
 \textheight  10 in
 \hoffset  0.1in

\usepackage{amsthm,amsmath,amssymb,amscd,verbatim,epsfig}
\usepackage{amsopn}
\usepackage{mathptmx}
\usepackage{amsfonts}
%\usepackage{setapace}
\usepackage{graphicx}
\usepackage{bm}
%\usepackage{CJK}
\usepackage{ulem}
\usepackage{multicol}
\usepackage{enumerate}
\usepackage{float}
\usepackage{fontspec}
\usepackage{xeCJK}
\setmainfont{Times New Roman}
\setCJKmainfont{TaipeiSansTCBeta-Regular}
\XeTeXlinebreaklocale "zh"
\XeTeXlinebreakskip = 0pt plus 1pt

\DeclareMathOperator{\closure}{cl}
\DeclareMathOperator{\interior}{int}

\title{Homework 7 of Introduction to Analysis (I), Honor Class}
\author{AM15 黃琦翔 111652028}

\begin{document}
\maketitle
\begin{enumerate}
    \item First, we show that $\displaystyle\bigcap_{i=1}^{\infty} F_i$ is closed.
    For $F_k$, we let $V_k = M\setminus F_k$ which is a open set since $F_k$ is compact(closed and bounded).
    Then, $M \setminus\displaystyle(\bigcap_{i=1}^{\infty} F_i)= \displaystyle\bigcup_{i=1}^{\infty} V_i = \displaystyle\lim_{i\to \infty} V_i$  is open.
    Thus, $\displaystyle\bigcap_{i=1}^{\infty} F_i$ is closed.

    Then, assume $\displaystyle\bigcap_{i=1}^{\infty} F_i$ is disconnected. 
    Thus, we can find two non-empty open sets $U, V$ in $\displaystyle\bigcap_{i=1}^{\infty} F_i$ s.t. $U\cap V = \emptyset$ and $\displaystyle\bigcap_{i=1}^{\infty} F_i = U \cup V$.
    Since $F_k \subseteq F_{k-1}$, $U, V \subseteq F_k$ for all k.

    And since $F_k$ is connected, $U \cup V \subsetneq F_k$ for all $k$.
    Then, we let $S_i  = F_i \setminus (U \cup V)$.
    Since $U, V$ are open, $U \cup V$ is open $\implies S_i$ is closed in $F_i$.

    Since $S_i\subseteq F_i \subseteq F_1$ and $F_1$ is conpact, by Nested Interval Property,
    $\displaystyle\bigcap_{i=1}^{\infty} S_i \neq \emptyset\implies \displaystyle\bigcap_{i=1}^{\infty} F_i = U \cup V \cup \displaystyle\bigcap_{i=1}^{\infty} S_i$ 
    contradict to $\displaystyle\bigcap_{i=1}^{\infty} F_i = U \cup V$.

    Thus, $\displaystyle\bigcap_{i=1}^{\infty} F_i$ is connected.

    \item  Assume there exists $a \in S$ and $r \geq 0$ s.t. $\lbrace x \mid d(x, a) = r\rbrace = \emptyset$.
    Then, $D(a, r) = \lbrace a \rbrace$ is open and $S \setminus \lbrace a \rbrace$ is open 
    since for all $x \in S\setminus\lbrace a \rbrace, D(x, d(x, a)) \subseteq S \setminus\lbrace a \rbrace$.
    We can find two open sets $D(a, r), S \setminus \lbrace a \rbrace$ s.t. $S$ is not connected.(contradiction)

    Thus, for every $a$ in $S$ and every $r > 0$, the set $\lbrace x : d(x, a) = r\rbrace$ is nonempty.

    \newpage
    \item Assume $B$ is disconnected, There are two non-empty open set $U, V\subseteq B$ s.t. $U \cup V = B$.
    Then, if $U \cap A$ and $V \cap A$ are relative open and nonempty in $A$ and disjoint in $A\implies A$ is not connected.(contradiction)

    If one of $U$ or $V$ is empty in $A$, that means some of accumulation point is not connected to $A$.
    Thus, we want to proof a accumulation point of open set $A$ is connected to $A$.

    Thus, $B$ is connected.

    \item$\ $\vspace{-20pt} \begin{enumerate}
        \item[($\implies$)] Assume $A$ is not in the set, that is, $A$ can be write as $\displaystyle\bigcup_{i \in E} I_i$ where $I_i$ is the interval in the set
        and $I_m, I_n$ is disconnected. Thus, $A$ is disconnected.

        therefore, $A$ is connected $\implies A$ is in the set.
      
        \item[($\impliedby$)]  For any $x, y$ in one of the four interval, 
        exists $f: [0, 1] \to [x, y]$ with $f(t) = tx + (1-t)y$ is a continuous path.
        Thus, $A$ is path-connected$\implies$ $A$ is connected.
    \end{enumerate}
\end{enumerate}
\end{document}
