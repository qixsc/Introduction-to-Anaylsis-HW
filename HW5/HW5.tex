\documentclass[12pt]{article}

\topmargin -40pt
\marginparwidth 0pt
\oddsidemargin  -40pt
\evensidemargin 0pt
\marginparsep 0pt
\linespread{1.9}
\textwidth 7.2 in
 \textheight  10 in
 \hoffset  0.1in

\usepackage{amsthm,amsmath,amssymb,amscd,verbatim,epsfig}
\usepackage{mathptmx}
\usepackage{amsfonts}
%\usepackage{setapace}
\usepackage{graphicx}
\usepackage{bm}
%\usepackage{CJK}
\usepackage{ulem}
\usepackage{multicol}
\usepackage{enumerate}
\usepackage{float}
\usepackage{fontspec}
\usepackage{xeCJK}
\setmainfont{Times New Roman}
\setCJKmainfont{TaipeiSansTCBeta-Regular}
\XeTeXlinebreaklocale "zh"
\XeTeXlinebreakskip = 0pt plus 1pt

\title{Homework 5 of Introduction to Analysis (I), Honor Class}
\author{AM15 黃琦翔 111652028}

\begin{document}
\maketitle
\begin{enumerate}
    \item First we want to show given $\epsilon > 0$, we can find finite sequence $x_1, x_2, \cdots x_n$ s.t. $\cup D(x_i, \epsilon) = X$.
    \begin{quote}
        Proving this by contrapositive way.

        Let $y_1$ be a point in $X$, then $D(y_1, \epsilon)$ can not cover $X$.
        Then, we can find $y_2\in X-D(y_1, \epsilon)$ s.t. $D(y_1, \epsilon) \cup D(y_2, \epsilon)$ can not cover $X$.

        Using the same way, we can find $y_m$ s.t. $\cup D(y_i, \epsilon)$ does not cover $X$. Which means distance of $y_i$ are greateer than $\epsilon$.
        This, is contradiction to the infinite set in $X$ has a accumulation points.
    \end{quote} 

    Thus, we can find finite set $\lbrace x_{i, n}\rbrace_{i=1}^{m_i}$ s.t. $\lbrace D(x_{i, n}, \dfrac{1}{n})\rbrace_{i=0}^{m_n}$ cover $X$.
    Then, $\lbrace x_{i, n}\rbrace_{1\leq n\leq \infty,\ 1\leq i\leq m_n}$ is countable.

    Then, we want to show $S= \cup D(x_{i, n}, \dfrac{1}{n})$ is dense in $X$.

    For any $x\in X-S$, then for all $\epsilon > 0$, exists $\dfrac{1}{n_i} \leq \epsilon$ s.t. $D(x, \dfrac{1}{n_i})\subseteq D(x, \epsilon)$ and
    $x \in D(x_{i, n}, \dfrac{1}{n_i})$ for some $1\leq i \leq m_i\implies x$ is a accumulation point of $S$. Then, $\bar{S} = X\implies S$ is dense.


    \item For any $U$ be any disjoint non-empty open set $\in \mathbb{R}$.
    Since $U$ is open, $U$ contains rational numbers.
    Then, for every $U$, we call it $U_x$, for a $x\in \mathbb{Q} \cap U$.
    Thus, we can label the open sets by $\mathbb{Q}$ which is countable.

    \newpage
    \item Since $A_1 \supseteq A_2\supseteq \cdots A_n$, $A_1' \supseteq A_2'\supseteq \cdots A_n'$.
    
    For $x \in A$, since $\cap_{n=1}^{\infty} A_n = \emptyset$, $A = \cap_{n=1}^{\infty} \bar{A}_n = (A_1\cup A_1') \cap (\cap_{n=2}^{\infty} \bar{A}_n)\\
    = (A_1' \cap (\cap_{n=2}^{\infty} \bar{A}_n)) \cup (A \cap (\cap_{n=2}^{\infty} \bar{A}_n)) = (\cap_{i=1}^{\infty} A_i') \cup (\cap_{i=2}^{\infty} A_i') \cup \cdots = \cap_{n=1}^{\infty} A_n'$.
    
    Thus, $x \in A_1'$.
    \item First, we want to proof $\overline{M-A} \supseteq (\overline{A} - A)$.
    Since $\overline{M-A} = (M-A)' \cup (M-A)$ and $\overline{A} \subseteq M\implies (\overline{A} - A) \subseteq (M - A) \subseteq \overline{M-A}$.
    
    $(A \cap \overline{M-A}) \cup (\overline{A} - A) = (A\cup (\overline{A}-A)) \cap (\overline{M-A} \cup (\overline{A} - A)) = \overline{A} \cap (\overline{M-A}) = \partial A$ 
\end{enumerate}
\end{document}
