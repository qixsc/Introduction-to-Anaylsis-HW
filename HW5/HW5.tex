\documentclass[12pt]{article}

\topmargin -40pt
\marginparwidth 0pt
\oddsidemargin  -40pt
\evensidemargin 0pt
\marginparsep 0pt
\linespread{1.5}
\textwidth 7.2 in
 \textheight  10 in
 \hoffset  0.1in

\usepackage{amsthm,amsmath,amssymb,amscd,verbatim,epsfig}
\usepackage{mathptmx}
\usepackage{amsfonts}
%\usepackage{setapace}
\usepackage{graphicx}
\usepackage{bm}
%\usepackage{CJK}
\usepackage{ulem}
\usepackage{multicol}
\usepackage{enumerate}
\usepackage{float}
\usepackage{fontspec}
\usepackage{xeCJK}
\setmainfont{Times New Roman}
\setCJKmainfont{TaipeiSansTCBeta-Regular}
\XeTeXlinebreaklocale "zh"
\XeTeXlinebreakskip = 0pt plus 1pt

\title{Homework 5 of Introduction to Analysis (I), Honor Class}
\author{AM15 黃琦翔 111652028}

\begin{document}
\maketitle
\begin{enumerate}
    \item Assume there exists uncountable dense subset in $X$.
    
    Since every infinite subset $E$ of $X$ have an accumulation points in $X$.
    Then, for $E$

    \item Assume there are uncountable collection of disjoint segments $\cup_I A_i = U$ where $U$ is a open set in $A$.
    Then, 

    \item For $x \in A$, since $\cap_{n=1}^{\infty} A_n = \emptyset$, $A = \cap_{n=1}^{\infty} \bar{A}_n = (\cap_{n=1}^{\infty} A_n) \cup (\cap_{n=1}^{\infty} A_n') = \cap_{n=1}^{\infty} A_n'$.
    
    Thus, $x \in A_1'$.

    \item First, we want to proof $\overline{M-A} \supseteq (\overline{A} - A)$.
    \begin{quote}
        Since $\overline{M-A} = (M-A)' \cup (M-A)$ and $\overline{A} \subseteq M\implies (\overline{A} - A) \subseteq (M - A) \subseteq \overline{M-A}$.
    \end{quote}
    
    $(A \cap \overline{M-A}) \cup (\overline{A} - A) = (A\cup (\overline{A}-A)) \cap (\overline{M-A} \cup (\overline{A} - A)) = \overline{A} \cap (\overline{M-A}) = \partial A$ 
\end{enumerate}
\end{document}
