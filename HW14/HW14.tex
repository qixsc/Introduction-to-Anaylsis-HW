\documentclass[12pt]{article}

\topmargin -40pt
\marginparwidth 0pt
\oddsidemargin  -40pt
\evensidemargin 0pt
\marginparsep 0pt
\linespread{1.9}
\textwidth 7.2 in
 \textheight  10 in
 \hoffset  0.1in

\usepackage{amsthm,amsmath,amssymb,amscd,verbatim,epsfig}
\usepackage{amsopn}
\usepackage{mathptmx}
\usepackage{amsfonts}
%\usepackage{setapace}
\usepackage{graphicx}
\usepackage{bm}
%\usepackage{CJK}
\usepackage{ulem}
\usepackage{multicol}
\usepackage{enumerate}
\usepackage{float}
\usepackage{fontspec}
\usepackage{xeCJK}
\setmainfont{Times New Roman}
\setCJKmainfont{TaipeiSansTCBeta-Regular}
\XeTeXlinebreaklocale "zh"
\XeTeXlinebreakskip = 0pt plus 1pt

\DeclareMathOperator{\closure}{cl}
\DeclareMathOperator{\interior}{int}

\title{Homework 13 of Introduction to Analysis (I), Honor Class}
\author{AM15 黃琦翔 111652028}

\begin{document}
\maketitle
\begin{enumerate}
    \item \begin{enumerate}
        \item Since $f(x) = \ln(x)$ is concave down, $\ln(uv) = \ln(u) + \ln(v) = \dfrac{1}{p}\ln(u^p) + \dfrac{1}{q}\ln(v^q) \leq \ln(\dfrac{u^p}{p} + \dfrac{v^q}{q})$.
        
        And since $\ln(x)$ is strictly increasing, $uv \leq \dfrac{u^p}{p} + \dfrac{v^q}{q}$.

        If $u^p = v^q$, $\dfrac{u^p}{p} + \dfrac{v^q}{q} = (\dfrac{1}{p} + \dfrac{1}{q}) u^p = u^p = u^{p(\frac{1}{p} + \frac{1}{q})} = u\cdot u^{\frac{p}{q}} = uv$.

        \item $\displaystyle\int_a^b \dfrac{(f(x))^p}{p} dx= \dfrac{1}{p}$ and $\displaystyle\int_a^b \dfrac{(g(x))^q}{q} = \dfrac{1}{q}$.
        
        Then, $\displaystyle\int_a^b \dfrac{(f(x))^p}{p} + \dfrac{(g(x))^q}{q} dx = \dfrac{1}{p} + \dfrac{1}{q} = 1$.

        And since $f(x)g(x) \leq \dfrac{(f(x)^p)}{p} + \dfrac{(g(x))^q}{q}$ for all $x\in [a, b]$ and $f, g$ are Reimann integral,
        $\displaystyle\int_a^b f(x)g(x) dx$ exists and $\displaystyle\int_a^b f(x) g(x) dx \leq \displaystyle\int_a^b \dfrac{(f(x))^p}{p} + \dfrac{(g(x))^q}{q} dx = 1$.

        \item 
    \end{enumerate}

    \item 
\end{enumerate}
\end{document}