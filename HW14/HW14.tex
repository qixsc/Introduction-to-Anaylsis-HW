\documentclass[12pt]{article}

\topmargin -40pt
\marginparwidth 0pt
\oddsidemargin  -40pt
\evensidemargin 0pt
\marginparsep 0pt
\linespread{1.9}
\textwidth 7.2 in
 \textheight  10 in
 \hoffset  0.1in

\usepackage{amsthm,amsmath,amssymb,amscd,verbatim,epsfig}
\usepackage{amsopn}
\usepackage{mathptmx}
\usepackage{amsfonts}
%\usepackage{setapace}
\usepackage{graphicx}
\usepackage{bm}
%\usepackage{CJK}
\usepackage{ulem}
\usepackage{multicol}
\usepackage{enumerate}
\usepackage{float}
\usepackage{fontspec}
\usepackage{xeCJK}
\setmainfont{Times New Roman}
\setCJKmainfont{TaipeiSansTCBeta-Regular}
\XeTeXlinebreaklocale "zh"
\XeTeXlinebreakskip = 0pt plus 1pt

\DeclareMathOperator{\closure}{cl}
\DeclareMathOperator{\interior}{int}

\title{Homework 13 of Introduction to Analysis (I), Honor Class}
\author{AM15 黃琦翔 111652028}

\begin{document}
\maketitle
\begin{enumerate}
    \item \begin{enumerate}
        \item Since $f(x) = \ln(x)$ is concave down, $\ln(uv) = \ln(u) + \ln(v) = \dfrac{1}{p}\ln(u^p) + \dfrac{1}{q}\ln(v^q) \leq \ln(\dfrac{u^p}{p} + \dfrac{v^q}{q})$.
        
        And since $\ln(x)$ is strictly increasing, $uv \leq \dfrac{u^p}{p} + \dfrac{v^q}{q}$.

        If $u^p = v^q$, $\dfrac{u^p}{p} + \dfrac{v^q}{q} = (\dfrac{1}{p} + \dfrac{1}{q}) u^p = u^p = u^{p(\frac{1}{p} + \frac{1}{q})} = u\cdot u^{\frac{p}{q}} = uv$.

        \item $\displaystyle\int_a^b \dfrac{(f(x))^p}{p} dx= \dfrac{1}{p}$ and $\displaystyle\int_a^b \dfrac{(g(x))^q}{q} = \dfrac{1}{q}$.
        Then, $\displaystyle\int_a^b \dfrac{(f(x))^p}{p} + \dfrac{(g(x))^q}{q} dx = \dfrac{1}{p} + \dfrac{1}{q} = 1$.

        And since $f(x)g(x) \leq \dfrac{(f(x)^p)}{p} + \dfrac{(g(x))^q}{q}$ for all $x\in [a, b]$ and $f, g$ are Reimann integral,
        $\displaystyle\int_a^b f(x)g(x) dx$ exists and $\displaystyle\int_a^b f(x) g(x) dx \leq \displaystyle\int_a^b \dfrac{(f(x))^p}{p} + \dfrac{(g(x))^q}{q} dx = 1$.

        \item Take $F = \displaystyle\int_a^b |f(x)|^p dx, \displaystyle\int_a^b |g(x)|^q dx$.
        
        Then, \begin{align*}
            \dfrac{|\int_a^b f(x)g(x) dx|}{F^{\frac{1}{p}}G^{\frac{1}{q}}} &\leq \dfrac{\int_a^b |f(x)||g(x)| dx}{F^\frac{1}{p}G^{\frac{1}{q}}}\\
            &= \int_a^b (\dfrac{|f(x)|^p}{F})^{\frac{1}{p}} (\dfrac{|g(x)|^p}{G})^{\frac{1}{q}} dx\\
            &\leq \int_a^b \dfrac{1}{p}(\dfrac{|f(x)|^p}{F})+ \dfrac{1}{q}(\dfrac{|g(x)|^q}{G}) dx\\
            &=\dfrac{1}{p} (\dfrac{\int_a^b |f(x)|^p dx}{F})^{\frac{1}{p}} + \dfrac{1}{q}(\dfrac{\int_a^b |g(x)|^q dx}{G})^{\frac{1}{q}}\\
            &= \dfrac{1}{p} = \dfrac{1}{q} = 1
        \end{align*}

        \newpage
        Thus, $|\displaystyle\int_a^b f(x) g(x) dx| \leq (\displaystyle\int_a^b |f(x)|^p dx)^{\frac{1}{p}} (\displaystyle\int_a^b |g(x)|^q dx)^{\frac{1}{q}}$.
    \end{enumerate}

    \item \begin{enumerate}
        \item Take $\delta = \dfrac{1}{2^{n+1}}$ and $P = \lbrace 0, \dfrac{1}{2^n}, \dfrac{1}{2^n}, \dfrac{1}{2^n} + \delta, \dfrac{1}{2^{n-1}}, \dfrac{1}{2^{n-1}} + \delta, \cdots, \dfrac{1}{2}+\delta, 1\rbrace$.

        Then, \begin{align*}
            U(f, P) - L(f, P) &= \delta \cdots (1 - \dfrac{1}{2} +\dfrac{1}{2} - \dfrac{1}{2^2} + \cdots + \dfrac{1}{2^{n-1}}-\dfrac{1}{2^n}) + \dfrac{1}{2^n}(\dfrac{1}{2^n} - 0)\\
            &= \dfrac{1}{2^{n+1}} \cdot \dfrac{1}{2^n} + \dfrac{1}{2^{2n}}\\
            &< \dfrac{1}{2^{2n-1}} \to 0\text{ as } n\to \infty
        \end{align*}

        \item Let $n = \left[-\dfrac{\ln x}{\ln 2}\right]$, then $A(x) = 2^{-n}$.
        \begin{align*}
            \int_0^x f(t) dt &= \int_0^1 f(t) dt - \int_x^1 f(t) dt\\
            &= \sum_{k=n}^{\infty} \dfrac{1}{2^k} \cdot \dfrac{1}{2^{k+1}} - (\dfrac{1}{2^{n}} - x) \cdot 2^{-n}\\
            &= x2^{-n} + \dfrac{1}{2^{2n+1}}\cdot \dfrac{4}{3} - \dfrac{1}{2^{2n}}\\
            &= x2^{-n} - \dfrac{1}{3}2^{-2n}\\
            &= xA(x) - \dfrac{1}{3}(A(x))^2
        \end{align*}
    \end{enumerate}
    
    \item Since $f$ is continuous on $[a, b]$, $f$ is uniform continuous and exists $c\in [a, b]$ s.t. $f(c) = M$.
    And let $I_n = (\displaystyle\int_a^b (|f(x)|)^n dx)^{\frac{1}{n}}$.
    
    Then, for all $\epsilon > 0$, there exists $\delta > 0$ s.t. if $x \in (c-\delta, c+\delta)\bigcap [a, b]$,
    $|f(x) - M| < \epsilon\implies f(x) \geq M - \epsilon$.

    Thus, $I_n \geq (\dfrac{1}{2\delta})^{\frac{1}{n}}(M-\epsilon)$.
    And since $f(x) \leq M$ for all $x$, $I_n \leq (\dfrac{1}{b-a})^\frac{1}{n} M$.

    Then, we can get for all $\epsilon > 0$ and some $\delta > 0$, $(\dfrac{1}{2\delta})^{\frac{1}{n}}(M-\epsilon) \leq I_n \leq (\dfrac{1}{b-a})^\frac{1}{n} M$.
    And since $r^{\frac{1}{n}} \to 1$ as $n \to \infty$ for all $r > 0$, $\displaystyle\lim_{n\to \infty} (\dfrac{1}{2\delta})^{\frac{1}{n}}(M-\epsilon) =M$ and $\displaystyle\lim_{n\to\infty} (\dfrac{1}{b-a})^\frac{1}{n} M = M$.
    
    By squeeze theorem, $\displaystyle\lim_{n\to\infty} I_n = M$.
\end{enumerate}
\end{document}