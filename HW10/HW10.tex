\documentclass[12pt]{article}

\topmargin -40pt
\marginparwidth 0pt
\oddsidemargin  -40pt
\evensidemargin 0pt
\marginparsep 0pt
\linespread{1.9}
\textwidth 7.2 in
 \textheight  10 in
 \hoffset  0.1in

\usepackage{amsthm,amsmath,amssymb,amscd,verbatim,epsfig}
\usepackage{amsopn}
\usepackage{mathptmx}
\usepackage{amsfonts}
%\usepackage{setapace}
\usepackage{graphicx}
\usepackage{bm}
%\usepackage{CJK}
\usepackage{ulem}
\usepackage{multicol}
\usepackage{enumerate}
\usepackage{float}
\usepackage{fontspec}
\usepackage{xeCJK}
\setmainfont{Times New Roman}
\setCJKmainfont{TaipeiSansTCBeta-Regular}
\XeTeXlinebreaklocale "zh"
\XeTeXlinebreakskip = 0pt plus 1pt

\DeclareMathOperator{\closure}{cl}
\DeclareMathOperator{\interior}{int}

\title{Homework 10 of Introduction to Analysis (I), Honor Class}
\author{AM15 黃琦翔 111652028}

\begin{document}
\maketitle
\begin{enumerate}
    \item \begin{enumerate}
        \item $\ $
        \begin{enumerate}
            \item[($\implies$)] Since $f$ is continuous, then for $x\in \bar{A}$, 
            we can find a $\lbrace x_i\mid x_i\in A\rbrace$ converges to $x$.
            Thus, for all $\epsilon > 0$, exists $\delta > 0$ s.t. $d(x_i, x) < \delta \implies \rho(f(x_i), f(x)) < \epsilon$.

            Then, for $y \in f(\bar{A})$, $D(y, \epsilon)\cap f(A) \neq \emptyset$ for all $\epsilon > 0$.
            Thus, $y \in \overline{f(A)}$.
            
            \item[($\impliedby$)] For $V$ closed in $T$, then $f(\closure(f^{-1}(V))) \subseteq \closure(V) = V$.
            Applied $f^{-1}$ on the both side, $\closure(f^{-1}(V)) \subseteq f^{-1}(\closure(V)) = f^{-1}(V)$.
            Thus, $f^{-1}(V)$ is closed if $V$ is closed implies $f$ is continuous.           

        \end{enumerate}

        \item Since $f(p) = g(p)$ for all $p \in S$ and $\bar{E} = S$.
        For any $x\in S$, $x\in \bar{E}$, then we can find a sequence $\lbrace x_i\rbrace\subseteq E$ converges to $x$.

        Since $f(x_i) = g(x_i)$ for all $i\in \mathbb{N}$, $|f(x) - g(x)| \leq |f(x) - f(x_i)| + |f(x_i) - g(x)| = |f(x) - f(x_i)| + |g(x)- g(x_i)| \to 0$ as $i\to \infty$.
        Then, $f(x) = g(x)$ for all $x\in S$.
    \end{enumerate}

    \item Since $B$ is compact, $f(B)$ is compact.
    Then, we want to show for an sequence $y_k$ converges to $y\in f(B)$,
    we can get $f^{-1}(y_k)$ converges to $f^{-1}(y)$.
    Since $f$ is one-to-one, we can find $x_k = f^{-1}(y_k)\in B$ and $x = f^{-1}(y)$.
    Then, assume $x_k$ is not converges to $x$, 
    $|x_k -x| > \delta$ for some $\delta > 0$ and $k\in \mathbb{N}$.
    Then, $f(D(x, \delta)\cap B) \not\subseteq D(f(x), \epsilon)$ since $f$ is one-to-one.
    Thus, $f$ is not continuous.(contradiction)

    Therefore, $f^{-1}$ is continuous.
    
    \newpage
    \item Since $f$ is strictly increasing, $f$ is one-to-one.
    By question $2$, $f^{-1}$ is continuous, too.

    Then, we want to check $f^{-1}$ is strictly increasing.
    For $f(x_1) < f(x_2)$ with $x_1, x_2 \in \mathbb{R}$.
    Assume $x_1 > x_2$.
    Then, $f(x_1) > f(x_2)$.(contradiction)

    Thus, $f(x_1) < f(x_2)\implies f^{-1}(f(x_1))=x_1 < x_2 = f^{-1}(f(x_2))$.
    Therefore, $f^{-1}$ is continuous and strictly increasing.
    
\end{enumerate}
\end{document}