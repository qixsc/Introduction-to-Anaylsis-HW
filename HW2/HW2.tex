\documentclass[12pt]{article}

\topmargin -40pt
\marginparwidth 0pt
 \oddsidemargin  -40pt
 \evensidemargin 0pt
 \marginparsep 0pt
\textwidth 7.2 in
 \textheight  10 in
 \hoffset  0.1in

\usepackage{amsthm,amsmath,amssymb,amscd,verbatim,epsfig}
\usepackage{mathptmx}
\usepackage{amsfonts}
%\usepackage{setapace}
\usepackage{graphicx}
\usepackage{bm}
%\usepackage{CJK}
\usepackage{ulem}
\usepackage{multicol}
\usepackage{enumerate}

\usepackage{fontspec}
\usepackage{xeCJK}
\setmainfont{Times New Roman}
\setCJKmainfont{TaipeiSansTCBeta-Regular}
\XeTeXlinebreaklocale "zh"
\XeTeXlinebreakskip = 0pt plus 1pt



\title{Homework 1 of Introduction to Analysis (I), Honor Class}
\author{AM15 黃琦翔 111652028}

\date{2023/9/19}

\begin{document}
\maketitle
\begin{enumerate}
    \item Without loss of generality, we take a sequence which is increasing. 
    Then, if $\lbrace x_n\rbrace$ is not Cauchy then there exists some $\epsilon > 0$ s.t. for any $N \in \mathbb{N}$, $n \geq m \geq N\Rightarrow x_n - x_m \geq \epsilon$.
    Then, we want to proof "not Cauchy implies not bounded"\begin{quote}
        if $\lbrace x_n\rbrace$ is bounded above by $y$, then choose $N = 1$ and $n_2 - n_1 > \epsilon$ implies $n_3 - n_2 > \epsilon$ and $n_4 - n_3 > \epsilon\Rightarrow n_k - n_1 > (k-1)\epsilon$.
        By Archimedean property, $(k-1)\epsilon > y$ for some $k$. Thus, $y$ is not upper bound of $\lbrace x_n \rbrace$ contradiction.
    \end{quote}
    That is, bounded and monotone sequence is Cauchy.
    For any bounded convergence sequence, it is Cauchy and converges in $\mathbb{R}$.
    Therefore, for any $\epsilon > 0$, $\exists N \in \mathbb{N}$ s.t. $|x - x_n| < \epsilon$ for some $n > N$.
    We want to proof $x$ is least upper bound of $\lbrace x_n\rbrace$.\begin{quote}
        For $x - \dfrac{1}{k}$, exists $0 < \epsilon < \dfrac{1}{k}\Rightarrow$ there exists some $n \in \mathbb{N}$ s.t. $x - x_n < \epsilon < \dfrac{1}{k}$.
        That is, $x_n > x-\dfrac{1}{k}$ for all $k \in \mathbb{N}$.

        Since $x_n$ is increasing, if there exists some $n \in\mathbb{N}$ s.t. $x_n > x$, 
        then there exists some $\epsilon > 0$ s.t. $|x_n -x| > \epsilon$ contradiction to the convergence.
        Thus, $x \geq x_n$ for all $n$.
    \end{quote}
    Therefore, $x$ is the least upper bound of the sequence.
    Thus, for any Cauchy sequence converges in $\mathbb{R}$, $\mathbb{R}$ has least upper bound property.

    \item \begin{enumerate}[(a)]
        \item If we can find infinite many points labeled by $n_1, n_2, \cdots$ s.t. $\lbrace x_{n_k}\rbrace_{k=1}^\infty$ is decreasing, thus $\lbrace x_n\rbrace_{n=1}^\infty$ have a decreasing subsequence.
        
        If not, then we only can find finite points s.t. $\lbrace x_{n_k}\rbrace_{k=1}^l$ is decreasing. Thus, we name the last element of the decreasing sequence as $N$.
        Thus, we can say that $n_1' = N + 1$, since $n_1'$ is not in decreasing sequence, implies that $\exists n_2' > n_1' \Rightarrow x_{n_2'} \geq x_{n_1'}$.
        And doing the same way,  we can find $n_1' < n_2' < n_3' < \cdots$ s.t. $\lbrace x_{n_i'}\rbrace_{i=1}^\infty$ is a increasing subsequence.

        Then, every sequence in $\mathbb{R}$ either has an increasing subsequence or a decreasing subsequence

        \item Since every sequence have either increasing or decreasing subsequence and $\lbrace x_n \rbrace$ is bounded,
        then by monotone convergence theorem, the subsequence converges.
    \end{enumerate} 
    
    \newpage
    \item \begin{align*}
        x_n = \dfrac{1}{n+1} + \dfrac{1}{n+2} + \cdots + &\dfrac{1}{2n}
        \notag
        \\x_{n+1} = \dfrac{1}{n+2} + \dfrac{1}{n+3} + \cdots + &\dfrac{1}{2n} + \dfrac{1}{2n+1} + \dfrac{1}{2n+2}
    \end{align*}

    Thus, $x_{n+1} - x_n = \dfrac{1}{2n+1} + \dfrac{1}{2n+2} - \dfrac{1}{n+1} = \dfrac{1}{2n+1} - \dfrac{1}{2n+2} = \dfrac{1}{(2n+1)(2n+2)}$.
    And $x_{n+k+1} - x_{n+k} = \dfrac{1}{(2(n+k)+1)(2(n+k)+2)} < \dfrac{1}{(2n+1)(2n+2)}$.

    Thus, for any $\epsilon > 0$, there exists $n, m \in \mathbb{N}, m>n$ s.t. $|x_{n} - x_{n+1}| = \dfrac{1}{(2n+1)(2n+1)} \leq \dfrac{\epsilon}{m-n}$.
    Then, $|x_{m} - x_n| \leq |x_m - x_{m-1}| + \cdots + |x_{n+1} - x_n| \leq (m-n) \dfrac{\epsilon}{m-n} = \epsilon$.

    Then, $\lbrace x_n\rbrace$ is Cauchy $\Rightarrow$ converges. 

    \item If $\displaystyle\liminf_{n \to \infty} x_n = \infty$, then assume $\lbrace x_n\rbrace$ is bounded above by $y$.
    Then, $x_n < y$ for all $n \in \mathbb{N}$ but $\liminf x_n = \infty > y$ contradiction.

    Thus, $\lbrace x_n \rbrace$ is not bounded above$\Rightarrow$ diverges to $\infty$.
\end{enumerate}
\end{document}
