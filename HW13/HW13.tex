\documentclass[12pt]{article}

\topmargin -40pt
\marginparwidth 0pt
\oddsidemargin  -40pt
\evensidemargin 0pt
\marginparsep 0pt
\linespread{1.9}
\textwidth 7.2 in
 \textheight  10 in
 \hoffset  0.1in

\usepackage{amsthm,amsmath,amssymb,amscd,verbatim,epsfig}
\usepackage{amsopn}
\usepackage{mathptmx}
\usepackage{amsfonts}
%\usepackage{setapace}
\usepackage{graphicx}
\usepackage{bm}
%\usepackage{CJK}
\usepackage{ulem}
\usepackage{multicol}
\usepackage{enumerate}
\usepackage{float}
\usepackage{fontspec}
\usepackage{xeCJK}
\setmainfont{Times New Roman}
\setCJKmainfont{TaipeiSansTCBeta-Regular}
\XeTeXlinebreaklocale "zh"
\XeTeXlinebreakskip = 0pt plus 1pt

\DeclareMathOperator{\closure}{cl}
\DeclareMathOperator{\interior}{int}

\title{Homework 13 of Introduction to Analysis (I), Honor Class}
\author{AM15 黃琦翔 111652028}

\begin{document}
\maketitle
\begin{enumerate}
    \item First, we want to check there are at most $\dfrac{n(n+1)}{2}$ points for $x\in [0, 1]$ for which $f(x) > \dfrac{1}{n}$.
    
    For $f(x) > \dfrac{1}{n}$, that means $x$ is a rational number $\dfrac{p}{q}$ and $q < n$.
    Thus, we have $x = \dfrac{1}{1},\ \dfrac{1}{2},\ \dfrac{1}{3},\ \dfrac{2}{3},\ \cdots$.
    The number of $x$ is less than $\dfrac{n(n+1)}{2}$.

    Then, we want to proof for any $\epsilon > 0$, we have a partition $P$ s.t. $|U(f, P) - L(f, p)| < \epsilon$.

    Then, for any $\epsilon > 0$, we take $P$ is partition of $[0, 1]$ which norms are $\dfrac{2}{n(n+1)}$.
    Since \begin{align*}
        |U(f, P) - L(f, P)| &= U(f, P)\\
        &< n \cdot \dfrac{2}{n(n+1)}\\
        &= \dfrac{2}{n+1}
    \end{align*}
    , for any $\epsilon > 0$, we can find $n$ s.t. $U(f, P) < \epsilon$.
    Thus, $f$ is integrable.


    \item Since $f$ is bounded, we can find $M\in \mathbb{R}$ s.t. $|f(x)| < M$ for all $x\in [a, b]$.
    Suppose there are $n$ points of discontinuity of $f$ on $[a, b]$.
    
    Thus, for any $\epsilon > 0$, we take $\delta = \dfrac{\epsilon}{4nM}$.
    Then, suppose the set of points of discontinuity is $\lbrace y_i\mid i \in \mathbb{N}, i \leq n\rbrace$.
    Take partition $P = \lbrace x_1 = y_1 - \epsilon,\ x_2 = y_1 + \epsilon,\ x_3 = y_2-\epsilon,\ \cdots\rbrace$ is finite.
    Then, for $f$ is continuous on $[a, x_1],\ [x_2, x_3],\ [x_4, x_5],\ \cdots$, 
    we only need to check other intervals $P'$ are Reimann integrable.
    $|U(f, P') - L(f, P')| \leq 2U(f, P') \leq 2\cdot n \cdot  (2\delta) \cdot M = 4nM \cdot \dfrac{\epsilon}{4nM} = \epsilon$.
    Thus, $f\in R([a, b])$.

    \item $f^{(n)}(x) = (\displaystyle\prod_{k=0}^{n-1} m-k) (1 + x)^{m-n}$
\end{enumerate}
\end{document}