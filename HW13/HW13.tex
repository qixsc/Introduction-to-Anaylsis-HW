\documentclass[12pt]{article}

\topmargin -40pt
\marginparwidth 0pt
\oddsidemargin  -40pt
\evensidemargin 0pt
\marginparsep 0pt
\linespread{1.9}
\textwidth 7.2 in
 \textheight  10 in
 \hoffset  0.1in

\usepackage{amsthm,amsmath,amssymb,amscd,verbatim,epsfig}
\usepackage{amsopn}
\usepackage{mathptmx}
\usepackage{amsfonts}
%\usepackage{setapace}
\usepackage{graphicx}
\usepackage{bm}
%\usepackage{CJK}
\usepackage{ulem}
\usepackage{multicol}
\usepackage{enumerate}
\usepackage{float}
\usepackage{fontspec}
\usepackage{xeCJK}
\setmainfont{Times New Roman}
\setCJKmainfont{TaipeiSansTCBeta-Regular}
\XeTeXlinebreaklocale "zh"
\XeTeXlinebreakskip = 0pt plus 1pt

\DeclareMathOperator{\closure}{cl}
\DeclareMathOperator{\interior}{int}

\title{Homework 13 of Introduction to Analysis (I), Honor Class}
\author{AM15 黃琦翔 111652028}

\begin{document}
\maketitle
\begin{enumerate}
    \item First, we want to check there are at most $\dfrac{n(n+1)}{2}$ points for $x\in [0, 1]$ for which $f(x) > \dfrac{1}{n}$.
    
    For $f(x) > \dfrac{1}{n}$, that means $x$ is a rational number $\dfrac{p}{q}$ and $q < n$.
    Thus, we have $x = \dfrac{1}{1},\ \dfrac{1}{2},\ \dfrac{1}{3},\ \dfrac{2}{3},\ \cdots$.
    The number of $x$ is less than $\dfrac{n(n+1)}{2}$. Thus, we suppose $a_n$ is the number of $x$ s.t.  $f(x) > \dfrac{1}{n}$

    For any $n\in \mathbb{N}$, we can find a $\delta > 0$ and  partition $P_n = \displaystyle\bigcup^n\ [x_i - \delta, x_i + \delta]$ s.t. $x_i$ is the only point $x\in [x_i-\delta, x_i + \delta]$ that $f(x) > \dfrac{1}{n}$  and $P_n' = [0, 1]\setminus P_n$.
    Thus, $|U(f, P_n) - L(f, P_n)| + |U(f, P_n') - L(f, P_n')| = U(f, p) + U(f, P_n') < a_n\cdot(2\delta\cdot 1) + (1-a_n\cdot 2\delta)\dfrac{1}{n} < \dfrac{1}{n} + a_n \cdot 2\delta$.
    Thus, we can find large enough $n$ and a tiny $\delta$ s.t. $f$ is integrable.

    \item Since $f$ is bounded, we can find $M\in \mathbb{R}$ s.t. $|f(x)| < M$ for all $x\in [a, b]$.
    Suppose there are $n$ points of discontinuity of $f$ on $[a, b]$.
    Thus, for any $\epsilon > 0$, we take $\delta = \dfrac{\epsilon}{8nM}$.
    Then, suppose the set of points of discontinuity is $\lbrace y_i\mid i \in \mathbb{N}, i \leq n\rbrace$.
    Take partition $P = \lbrace x_1 = y_1 - \epsilon,\ x_2 = y_1 + \epsilon,\ x_3 = y_2-\epsilon,\ \cdots\rbrace$ is finite.
    Then, for $f$ is continuous on $[a, x_1],\ [x_2, x_3],\ [x_4, x_5],\ \cdots$, 
    we can find a partition $P_1$ s.t. $|U(f, P_1) - L(f, P_2)| < \dfrac{\epsilon}{2}$
    we only need to check other intervals $P_2 = [x_1, x_2]\bigcup [x_3, x_4]\bigcup \cdots \bigcup [x_{2n-1}, x_{2n}]$ are Reimann integrable.
    $|U(f, P_2) - L(f, P_2)| \leq 2U(f, P_2) \leq 2\cdot n \cdot  (2\delta) \cdot M = 4nM \cdot \dfrac{\epsilon}{8nM} = \dfrac{\epsilon}{2}$.
    Thus, $|U(f, P) - L(f, P)| \leq |U(f, P_1) - L(f, P_1)| + |U(f, P_2), L(f, P_2)| < \dfrac{\epsilon}{2} + \dfrac{\epsilon}{2} = \epsilon$, which implies that $f\in R([a, b])$.

    \newpage
    \item $f^{(n)}(x) = (\displaystyle\prod_{k=0}^{n-1} m-k) (1 + x)^{m-n}$.
    $R_n = x^n \dfrac{f^{(n)}(\theta_n x)}{n!} = (\displaystyle\prod_{k=1}^{n}\dfrac{m-k+1}{k}) (1+\theta_n x)^{m-n}x^n$.

    For $m > 0$, we can find $N\in \mathbb{N}$ s.t. $N > m$ and $N-1 \leq m$.
    Since $x > 0$, $1 + \theta_n x > 1$, $(1+\theta_n x)^{m-n} < 1$ for $n > N$.
    And since $n > N > M$, $\displaystyle\prod_{k=1}^{n}\dfrac{m-k+1}{k} = \dfrac{m}{N} \cdot \dfrac{m-1}{N-1} \cdots \dfrac{m-N+1}{1} \cdot \dfrac{m-N}{N+1}\cdots < 1$.
    Thus, $R_n < x^n$ for $m > 0$ and $n > m$.
    
    For $m < 0$, since $|\dfrac{m-k}{k}| > |\dfrac{m-k-1}{k+1}| > \cdots > 1$,
    there exists $N\in \mathbb{N}$ s.t. $|\dfrac{m-n}{n}| < \dfrac{1}{x}$ for all $n > N$.
    Therefore, we can find $l < 1$ s.t. $|\dfrac{m-n}{n}x| < l$ for all $n > N$.
    Thus, $|R_n| < |\displaystyle\prod_{k=1}^{n} (\dfrac{m-k+1}{k}x)| = |A|\cdot |\displaystyle\prod_{k=N}^{^n} \dfrac{m-k+1}{k}x| < |A|\cdot l^{n-N} \to 0$ as $n\to\infty$.

    If $x < 0$, $(1 + \theta x) < 1\implies (1 +\theta x) > 1$.
    Thus, we can't use the same argument to prove.
\end{enumerate}
\end{document}