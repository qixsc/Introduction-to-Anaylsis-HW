\documentclass[12pt]{article}

\topmargin -40pt
\marginparwidth 0pt
\oddsidemargin  -40pt
\evensidemargin 0pt
\marginparsep 0pt
\linespread{1.9}
\textwidth 7.2 in
 \textheight  10 in
 \hoffset  0.1in

\usepackage{amsthm,amsmath,amssymb,amscd,verbatim,epsfig}
\usepackage{amsopn}
\usepackage{mathptmx}
\usepackage{amsfonts}
%\usepackage{setapace}
\usepackage{graphicx}
\usepackage{bm}
%\usepackage{CJK}
\usepackage{ulem}
\usepackage{multicol}
\usepackage{enumerate}
\usepackage{float}
\usepackage{fontspec}
\usepackage{xeCJK}
\setmainfont{Times New Roman}
\setCJKmainfont{TaipeiSansTCBeta-Regular}
\XeTeXlinebreaklocale "zh"
\XeTeXlinebreakskip = 0pt plus 1pt

\DeclareMathOperator{\closure}{cl}
\DeclareMathOperator{\interior}{int}

\title{Homework 11 of Introduction to Analysis (I), Honor Class}
\author{AM15 黃琦翔 111652028}

\begin{document}
\maketitle
\begin{enumerate}
    \item \begin{enumerate}
        \item Let $g(x_0) = \displaystyle\lim_{x\to x_0} f(x)$ for all $x_0 \in \bar{A}$.
        Since $f$ is uniform continuous, $g(x) = f(x)$ for all $x\in A$ and $g$ is continuous.
        Then, we want to show $g$ is uniform continuous.

        Assume $g$ is not uniform continuous, exists a $\epsilon > 0$, for all $\delta > 0$ 
        s.t. $|x - x_0| < \delta$ and $|g(x) - g(x_0)| > \epsilon$.

        \item Since $A$ is bounded, $\bar{A}$ is compact.
        Then, $f(A) \subseteq f(\bar{A})$ is bounded since $f(\bar{A})$ is compact.
    \end{enumerate}

    \item \begin{enumerate}
        \item For any $x_0\in \mathbb{R}^n$ and $\epsilon > 0$, $f_A(x_0) - \epsilon < f_A(x) < f_A(x_0) + \epsilon$ for $x\in D(x_0, \epsilon)$ tirvially.
        Then, $|f_A(x) - f_A(x_0)| < \delta = \epsilon$ for $\| x - x_0 \| < \epsilon$.
        Thus, $f_A$ is uniform continuous.

        \item For $x\in \bar{A}$, $\inf\{ \| x-y\| \mid y\in A\} = 0$.
        Thus, $f_A(x) = 0$ for all $x\in \bar{A}$

        And since $x\notin \bar{A}$, $\inf\{ \| x-y\| \mid y\in A\} > 0$, $\bar{A} = \{ x\mid f_A(x) = 0\}$
    \end{enumerate}
    
    \item Since $f$ is one-to-one, then $f(x) = f(y)$ implies $x = y$.
    Assume $f$ is not monotonic but continuous, exists a $x$ s.t. $x < x'$ for all $x' \in D(x, \epsilon)$ 
    or $x > x'$ for all $x' \in D(x, \epsilon)$.
    Then, since $(a, b)$ is path-connected, by IVT, we can find $y, y'\in D(x, \epsilon)$ s.t. $f(y) = f(y')$ with $y \neq y'$.
    That is, $f$ is not one-to-one(contradiction).

    Thus, $f$ is one-to-one and monotonic.
\end{enumerate}
\end{document}