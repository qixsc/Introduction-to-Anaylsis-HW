\documentclass[12pt]{article}

\topmargin -40pt
\marginparwidth 0pt
\oddsidemargin  -40pt
\evensidemargin 0pt
\marginparsep 0pt
\linespread{1.9}
\textwidth 7.2 in
 \textheight  10 in
 \hoffset  0.1in

\usepackage{amsthm,amsmath,amssymb,amscd,verbatim,epsfig}
\usepackage{amsopn}
\usepackage{mathptmx}
\usepackage{amsfonts}
%\usepackage{setapace}
\usepackage{graphicx}
\usepackage{bm}
%\usepackage{CJK}
\usepackage{ulem}
\usepackage{multicol}
\usepackage{enumerate}
\usepackage{float}
\usepackage{fontspec}
\usepackage{xeCJK}
\setmainfont{Times New Roman}
\setCJKmainfont{TaipeiSansTCBeta-Regular}
\XeTeXlinebreaklocale "zh"
\XeTeXlinebreakskip = 0pt plus 1pt

\DeclareMathOperator{\closure}{cl}
\DeclareMathOperator{\interior}{int}

\title{Homework 8 of Introduction to Analysis (I), Honor Class}
\author{AM15 黃琦翔 111652028}

\begin{document}
\maketitle
\begin{enumerate}
    \item\begin{enumerate}
        \item Suppose $\displaystyle\prod_{n=1}^{\infty} a_n = a \in \mathbb{R}^+$, 
        $\exp(\displaystyle\sum_{n=1}^{\infty} \ln(a_n)) = \displaystyle\prod_{n=1}^{\infty} a_n = a$.
        Thus, $\displaystyle\sum_{n=1}^{\infty} ln(a_n) = \ln(a)\in (-\infty, \infty)$

        Therefore, $\displaystyle\prod_{n=1}^{\infty} a_n$ converges iff $\displaystyle\sum_{n=1}^{\infty} \ln(a_n)$ converges.

        \item Suppose $\displaystyle\sum_{n=1}^{\infty} u_n$ convergs.
        $\displaystyle\prod_{n=1}^{\infty} (1 + u_n) = \displaystyle\sum_{n=1}^{\infty} \ln(1 + u_n)$
    \end{enumerate}

    \item

    Counter example: $a_n = $, then $\displaystyle\sum_{n=1}^{\infty}\sqrt{\dfrac{a_n}{n}} = $
\end{enumerate}
\end{document}
