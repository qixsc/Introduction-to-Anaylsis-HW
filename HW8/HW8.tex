\documentclass[12pt]{article}

\topmargin -40pt
\marginparwidth 0pt
\oddsidemargin  -40pt
\evensidemargin 0pt
\marginparsep 0pt
\linespread{1.9}
\textwidth 7.2 in
 \textheight  10 in
 \hoffset  0.1in

\usepackage{amsthm,amsmath,amssymb,amscd,verbatim,epsfig}
\usepackage{amsopn}
\usepackage{mathptmx}
\usepackage{amsfonts}
%\usepackage{setapace}
\usepackage{graphicx}
\usepackage{bm}
%\usepackage{CJK}
\usepackage{ulem}
\usepackage{multicol}
\usepackage{enumerate}
\usepackage{float}
\usepackage{fontspec}
\usepackage{xeCJK}
\setmainfont{Times New Roman}
\setCJKmainfont{TaipeiSansTCBeta-Regular}
\XeTeXlinebreaklocale "zh"
\XeTeXlinebreakskip = 0pt plus 1pt

\DeclareMathOperator{\closure}{cl}
\DeclareMathOperator{\interior}{int}

\title{Homework 8 of Introduction to Analysis (I), Honor Class}
\author{AM15 黃琦翔 111652028}

\begin{document}
\maketitle
\begin{enumerate}
    \item\begin{enumerate}
        \item Suppose $\displaystyle\prod_{n=1}^{\infty} a_n = a \in \mathbb{R}^+$, 
        $\exp(\displaystyle\sum_{n=1}^{\infty} \ln(a_n)) = \displaystyle\prod_{n=1}^{\infty} a_n = a$.
        Thus, $\displaystyle\sum_{n=1}^{\infty} ln(a_n) = \ln(a)\in (-\infty, \infty)$

        Therefore, $\displaystyle\prod_{n=1}^{\infty} a_n$ converges iff $\displaystyle\sum_{n=1}^{\infty} \ln(a_n)$ converges.

        \item Since $u_n\geq 0$ and converges, $\displaystyle\sum_{n=1}^{\infty} u_n$ converges if $u_n$ converges to $0$.
        By limit comparison test $\displaystyle\lim_{n\to\infty} \dfrac{\ln(1 + u_n)}{u_n} = \displaystyle\lim_{x\to 0} \dfrac{\ln(1+x)}{x} = 1$,
        $u_n$ converges $\iff$ $\displaystyle\prod_{n=1}^{\infty} (1 + u_n)$ converges.

        \item Since $\displaystyle\sum_{n=1}^{\infty}u_n$ is absolutely convergent, by (b), $\displaystyle\prod_{n=1}^{\infty} (1 + |u_n|)$ is converges.
        For any $\epsilon > 0$, $\exists N \in \mathbb{N}$, for all $a, b> N$ s.t. $|\displaystyle\sum_{n=1}^{a} \ln(1+|u_n|) - \displaystyle\sum_{n=1}^{b}\ln(1 + |u_n|)| < \epsilon$.
        Then, $|\displaystyle\sum_{n=1}^{a} \ln(1 + u_n) - \displaystyle\sum_{n=1}^{b} \ln(1 + u_n)| < \epsilon$. Which implies $\displaystyle\sum_{n=1}^{\infty} \ln(1 + u_n)$ converges.
        Thus, $\displaystyle\prod_{n=1}^{\infty} (1 + u_n)$ converges.
    \end{enumerate}

    \item $\dfrac{\sqrt{a_n}}{n^p}  = \sqrt{\dfrac{a_n}{n^{2p}}} \leq \dfrac{a_n + n^{-2p}}{2}$ by AM-GM Inequality.
    Then, if $p\leq \dfrac{1}{2}$, then $\dfrac{1}{n^{2p}} \geq \dfrac{1}{n}$ diverges.
    Therefore, if $p > \dfrac{1}{2}$, $\displaystyle\sum_{n=1}^{\infty} \sqrt{a_n} \cdot n^{-p} \leq \displaystyle\sum_{n=1}^{\infty} (\dfrac{a_n + n^{-2p}}{2})$ converges by p-test.

    Counter example: $a_n = \dfrac{1}{n (\ln(n))^2}$, then $\displaystyle\sum_{n=1}^{\infty}\sqrt{\dfrac{a_n}{n}} = \displaystyle\sum_{n=1}^{\infty} \dfrac{1}{n(\ln(n))}$ diverges by integral-test. 
    \end{enumerate}
\end{document}
