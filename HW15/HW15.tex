\documentclass[12pt]{article}

\topmargin -40pt
\marginparwidth 0pt
\oddsidemargin  -40pt
\evensidemargin 0pt
\marginparsep 0pt
\linespread{1.9}
\textwidth 7.2 in
 \textheight  10 in
 \hoffset  0.1in

\usepackage{amsthm,amsmath,amssymb,amscd,verbatim,epsfig}
\usepackage{amsopn}
\usepackage{mathptmx}
\usepackage{amsfonts}
%\usepackage{setapace}
\usepackage{graphicx}
\usepackage{bm}
%\usepackage{CJK}
\usepackage{ulem}
\usepackage{multicol}
\usepackage{enumerate}
\usepackage{float}
\usepackage{fontspec}
\usepackage{xeCJK}
\setmainfont{Times New Roman}
\setCJKmainfont{TaipeiSansTCBeta-Regular}
\XeTeXlinebreaklocale "zh"
\XeTeXlinebreakskip = 0pt plus 1pt

\DeclareMathOperator{\closure}{cl}
\DeclareMathOperator{\interior}{int}

\title{Homework 15 of Introduction to Analysis (I), Honor Class}
\author{AM15 黃琦翔 111652028}

\begin{document}
\maketitle
\begin{enumerate}
    \item \begin{enumerate}
        \item For any $y\in [c, d]$, we can find a partition $P_y = $

        \item $F'(y) = \displaystyle\lim_{h\to 0} \dfrac{F(y+h) - F(y)}{h} = \displaystyle\int_a^b \displaystyle\lim_{h\to 0} \dfrac{f(x, y+h) - f(x, y)}{h} dx = \displaystyle\int_a^b \dfrac{\partial f}{\partial y}(x, y) dx$.
    \end{enumerate}

    \item\begin{enumerate}
        \item $g'(x) = \displaystyle\int_0^1 -2(t^2+1)x\dfrac{e^{-x^2(t^2 + 1)}}{t^2+1}dt = \displaystyle\int_0^1 -2xe^{-x^2(t^2+1)} dt$.
        
        $f'(x) = 2e^{-x^2}(\displaystyle\int_0^x e^{-t^2} dt)$

        Since $f'(x) + g'(x) = 0$ for all $x$, $f(x) + g(x) = f(0) + g(0) = g(0) = \displaystyle\int_0^1 \dfrac{1}{t^2 + 1} dt =\displaystyle\int_0^\frac{\pi}{4} du = \dfrac{\pi}{4}$.

        \item Since $g(x) \to 0$ as $x \to \infty$, $f(x) \to \dfrac{\pi}{4}$.
        Then, $\displaystyle\lim_{x\to\infty} \displaystyle\int_0^x e^{-t^2} dt = \displaystyle\lim_{x\to\infty} \sqrt{f(x)} = \sqrt{\dfrac{\pi}{4}} = \dfrac{\sqrt{\pi}}{2}$.
    \end{enumerate}

    \item \begin{enumerate}
        \item If $L = \lim_{x \to \infty} f(x) > 0$, we can find a $N\in\mathbb{N}$ s.t. $f(x) > \dfrac{L}{2}$ for all $x > N$.
        
        Then, for $t > N$, $\displaystyle\int_0^t f(x) dx = \displaystyle\int_0^N f(x) dx + (t-N)\dfrac{L}{2}$.
        Thus, $\displaystyle\lim_{t\to\infty} \displaystyle\int_0^t f(x) dx = \infty$ doesn't exists(contradiction).

        Using the same way, $L < 0$ also causes contradiction.
        Therefore, $L = 0$.

        \item First, for any $0<\epsilon < 1$, we can not find a $N \in \mathbb{N}$ s.t. $f(x) < \epsilon$ for all $x > N$.
        
        Then, let $n = [x]$, $\displaystyle\int_0^x f(t) dt \leq \displaystyle\sum_{i=1}^{n} 1 \cdot 2^{-i}$.

        Since $\displaystyle\sum_{i=1}^{\infty} 2^{-i} = 1 < \infty$, $\displaystyle\lim_{x \to \infty}\displaystyle\int_0^x f(t) dt \leq 1$, $f$ is improperly integrable.
    \end{enumerate}
\end{enumerate}
\end{document}