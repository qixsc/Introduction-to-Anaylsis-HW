\documentclass[12pt]{article}

\topmargin -40pt
\marginparwidth 0pt
 \oddsidemargin  -40pt
 \evensidemargin 0pt
 \marginparsep 0pt
\textwidth 7.2 in
 \textheight  10 in
 \hoffset  0.1in

\usepackage{amsthm,amsmath,amssymb,amscd,verbatim,epsfig}
\usepackage{mathptmx}
\usepackage{amsfonts}
%\usepackage{setapace}
\usepackage{graphicx}
\usepackage{bm}
%\usepackage{CJK}
\usepackage{ulem}
\usepackage{multicol}
\usepackage{enumerate}

\usepackage{fontspec}
\usepackage{xeCJK}
\setmainfont{Times New Roman}
\setCJKmainfont{TaipeiSansTCBeta-Regular}
\XeTeXlinebreaklocale "zh"
\XeTeXlinebreakskip = 0pt plus 1pt




\title{Homework 1 of Introduction to Analysis (I), Honor Class}
\author{AM15 黃琦翔 111652028}

\date{2023/9/12}

\begin{document}
\maketitle
\begin{enumerate}
    \item Let $R = \lbrace f(x) \mid x \in X \rbrace$ and $r = \sup R$. 
    Then, for any $\beta$ which is upper bound of $R$, $r \leq \beta$. 
    Thus, $a + \beta$ is an upper bound of $R' = \lbrace a + f(x) \mid x \in X \rbrace$.

    Assume there exists $\beta' = \sup R'$ s.t. $\beta' < a+r$. 
    Since $\beta' \geq a + f(x),\ \forall x \in X \Rightarrow \beta' - a \geq f(x),\ \forall x \in X$. 
    Therefore, $\beta' - a$ is an upper bound of $R$ and it should greater or equal to $r$. 
    This causes contradiction to the assumption.

    Thus, for every upper bound f $R'$, it is greater or equal to $a + r$.
    
    That is $\sup\lbrace a + f(x) \mid x \in X \rbrace = a + \lbrace f(x) \mid x \in X \rbrace$

    \item let $a$ be greastest lower bound of $\lbrace f(x) \mid x \in X\rbrace$, $b$ be greastest lower bound of $\lbrace f(x) \mid x \in X\rbrace$.
    That is $a \leq f(x) \forall x \in X$ and $b \leq g(x) \forall x \in X$.
    Then, $a + b$ is an lower bound of $\lbrace f(x) + g(x) \mid x \in X \rbrace$. Therefore $a + b \leq \inf \lbrace f(x) + g(x) \mid x \in X\rbrace$.
    In other world, $\inf \lbrace f(x) \mid x \in X\rbrace + \inf \lbrace g(x) \mid x \in X\rbrace \leq \inf \lbrace f(x) + g(x) \mid x \in X\rbrace$

    Using the same way we can prove $\sup \lbrace f(x) + g(x) \mid x \in X \rbrace \leq \sup \lbrace f(x) \mid x \in X \rbrace + \sup \lbrace g(x) \mid x \in X \rbrace$

    Then, we want to show $\inf S \leq \sup S$ for all set $S$:
    \begin{quote}
        For a upper bound of $S$ "$\beta$" and lower bound of $S$ "$\alpha$". For any element $x \in S$, $\alpha \leq x \leq \beta$.

        Then, $\inf S \leq x \leq \sup S$ for all $x\in S$
    \end{quote}

    Then, $\inf \lbrace f(x) \mid x \in X \rbrace + \inf \lbrace g(x) \mid x \in X \rbrace \leq \inf \lbrace f(x) + g(x) \mid x \in X\rbrace \leq \sup \lbrace f(x) + g(x) \mid x \in X \rbrace \leq \sup \lbrace f(x) \mid x \in X \rbrace + \sup \lbrace g(x) \mid x \in X \rbrace$
    
    Ex. Let $f(x) = \sin(x),\ g(x) = \cos(x),\ X = (0, 2\pi)$, then $-2 < -\sqrt{2} < 2$.

    \newpage

    \item Using the steps from Rudin's book.
    \begin{enumerate}[(a)]
        \item For any $n \in \mathbb{N}$, $b^n -1 \geq n(b-1)$
            \begin{quote}
            {\bf Proof.}\begin{itemize}
                \item $n = 1$, $b -1 \geq 1(b-1)$ trivially true
                \item Assume $n = k$, $b^k-1 \geq k(b-1)$
                \item $n = k+1$, $b^{k+1} -1 = (b-1)(b^k + b^{k-1} + \cdots + 1) \geq (b-1)(k+1)$ since $b \geq 1$
            \end{itemize}
            Then, by math induction, $b^n-1 \geq n(b-1)$ for all $n \in \mathbb{N}$
        \end{quote}
        \item Then, $b-1 \geq n(b^{1/n} - 1)$\begin{quote}
            {\bf Proof.}$(b^{1/n})^n -1 \geq n(b^{1/n} -1)$
        \end{quote}
        \item If $t > 1$ and $n > \dfrac{b-1}{t-1}$, then $b^{1/n} < t$\begin{quote}
            {\bf Proof.} $n > \dfrac{b-1}{t-1} \geq \dfrac{n(b^{1/n} -1)}{t-1}$, then $t-1 > b^{1/n} -1 \Rightarrow b^{1/n} < t$ for $t > 1$
        \end{quote}
        \item If $w$ s.t. $b^w < y$, then exists some $n \in \mathbb{N}$ s.t. $b^{w + 1/n} < y$\begin{quote}
            {\bf Proof.} $b^{w + 1/n} = b^w *b^{1/n}$, then from (c), we can know that there exists some $n \in \mathbb{N}$ s.t. $b^{1/n} < \dfrac{y}{b^w}$ since $\dfrac{y}{b^w} > 1$. Thus, $b^{w + 1/n} < y$ if $b^w < y$ with some $n \in \mathbb{N}$
        \end{quote}
        \item If $b^w > y$, then $b^{w - 1/n} > y$\begin{quote}
            {\bf Proof.} $b^{w - 1/n} = \dfrac{b^w}{b^1/n}$. By (c), there exists some $n \in \mathbb{N}$ s.t. $b^{1/n} \leq \dfrac{b^w}{y}$. Thus, $b^{w-1/n} > y$ if $b^w > y$ for some $n \in \mathbb{N}$
        \end{quote}
        \item $A = \lbrace w \mid b^w < y\rbrace$, $x = sup A$ satisfies $b^x = y$\begin{quote}
            By Bernoulli's inequality, $b^n = (1 + c)^n \geq 1 + nc$ by let $c = b-1 > 0$, then by Archimedean property, there exists some $n \in \mathbb{N}$ s.t. $b^n > 1 + nc > y$. Thus, $A$ is bounded above.
            And since $b^{-n} = \dfrac{1}{b^n}$, there exists some $n \in \mathbb{N}$ s.t. $b^n \geq \dfrac{1}{y}\Rightarrow b^{-n} \leq y$. Thus, $A \neq \emptyset$.

            Thus, by the completement of real number, there exists $x = \sup A$.

            If $b^x < y$, from (d), $b^{x + 1/n} < y\Rightarrow x + 1/n \in A$. Thus, $x$ is not upper bound of $A$.
            If $b^x > y$, form (f), $b^{x-1/n} > y$. Then, $x$ is not the least upper bound of $A$.

            So, $b^x$ must to be equal to $y$.
        \end{quote}
        \item $x$ is unique because it is supremun of $A$.
    \end{enumerate}
\end{enumerate}
\end{document}
