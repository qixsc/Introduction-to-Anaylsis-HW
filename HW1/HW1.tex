\documentclass[12pt]{article}


\topmargin -40pt
\marginparwidth 0pt
 \oddsidemargin  -40pt
 \evensidemargin 0pt
 \marginparsep 0pt
\textwidth 7.2 in
 \textheight  10 in
 \hoffset  0.1in

\usepackage{amsthm,amsmath,amssymb,amscd,verbatim,epsfig}
\usepackage{mathptmx}
\usepackage{amsfonts}
%\usepackage{setapace}
\usepackage{graphicx}
\usepackage{bm}
%\usepackage{CJK}
\usepackage{ulem}
\usepackage{multicol}
\usepackage{enumerate}

\usepackage{fontspec}
\usepackage{xeCJK}
\setmainfont{Times New Roman}
\setCJKmainfont{TaipeiSansTCBeta-Regular}
\XeTeXlinebreaklocale "zh"
\XeTeXlinebreakskip = 0pt plus 1pt




\title{Homework 1 of Introduction to Analysis (I), Honor Class}
\author{AM15 黃琦翔 111652028}

\date{2023/9/12}

\begin{document}
\maketitle
\begin{enumerate}
    \item Let $R = \lbrace f(x) \mid x \in X \rbrace$ and $r = \sup R$. 
    Then, for any $\beta$ which is upper bound of $R$, $r \leq \beta$. 
    Thus, $a + \beta$ is an upper bound of $R' = \lbrace a + f(x) \mid x \in X \rbrace$.

    Assume there exists $\beta' = \sup R'$ s.t. $\beta' < a+r$. 
    Since $\beta' \geq a + f(x),\ \forall x \in X \Rightarrow \beta' - a \geq f(x),\ \forall x \in X$. 
    Therefore, $\beta' - a$ is an upper bound of $R$ and it should greater or equal to $r$. 
    This causes contradiction to the assumption.

    Thus, for every upper bound f $R'$, it is greater or equal to $a + r$.
    
    That is $\sup\lbrace a + f(x) \mid x \in X \rbrace = a + \lbrace f(x) \mid x \in X \rbrace$

    \item let $a$ be greastest lower bound of $\lbrace f(x) \mid x \in X\rbrace$, $b$ be greastest lower bound of $\lbrace f(x) \mid x \in X\rbrace$.
    That is $a \leq f(x) \forall x \in X$ and $b \leq g(x) \forall x \in X$.
    Then, $a + b$ is an lower bound of $\lbrace f(x) + g(x) \mid x \in X \rbrace$. Therefore $a + b \leq \inf \lbrace f(x) + g(x) \mid x \in X\rbrace$.
    In other world, $\inf \lbrace f(x) \mid x \in X\rbrace + \inf \lbrace g(x) \mid x \in X\rbrace \leq \inf \lbrace f(x) + g(x) \mid x \in X\rbrace$

    Using the same way we can prove $\sup \lbrace f(x) + g(x) \mid x \in X \rbrace \leq \sup \lbrace f(x) \mid x \in X \rbrace + \sup \lbrace g(x) \mid x \in X \rbrace$

    Then, we want to show $\inf S \leq \sup S$ for all set $S$:
    \begin{quote}
        For a upper bound of $S$ "$\beta$" and lower bound of $S$ "$\alpha$". For any element $x \in S$, $\alpha \leq x \leq \beta$.

        Then, $\inf S \leq x \leq \sup S$ for all $x\in S$
    \end{quote}

    Then, $\inf \lbrace f(x) \mid x \in X \rbrace + \inf \lbrace g(x) \mid x \in X \rbrace \leq \inf \lbrace f(x) + g(x) \mid x \in X\rbrace \leq \sup \lbrace f(x) + g(x) \mid x \in X \rbrace \leq \sup \lbrace f(x) \mid x \in X \rbrace + \sup \lbrace g(x) \mid x \in X \rbrace$
    
    Ex. Let $f(x) = \sin(x),\ g(x) = \cos(x),\ X = (0, 2\pi)$, then $-2 < -\sqrt{2} < 2$.

    \newpage

    \item Let $A = \lbrace x \mid x \in \mathbb{R},\ b^x < y\rbrace$.
    
    By Bernoulli's Inequality(In Bartle's book page.35) and let $b = 1+c$ with $c \in \mathbb{R},\ c > 0$, then $(1+c)^n \geq 1 + nc$ with all $n \in N$.
    Then, we can find some $n \in \mathbb{N}$ s.t. $c > \dfrac{y}{n}$. Therefore, $b^n = (1+c)^n \geq 1+nc > y$ and $n$ is upper bound of $A$. Thus, by completeness of $\mathbb{R}$, there exists $\beta \in \mathbb{R}$ be $\sup A$. 
    
    If $b^\beta < y$, then assume $\beta + \dfrac{1}{n} \in A$. Thus $b^{\beta + \frac{1}{n}} = b^\beta \cdot b^\frac{1}{n} > b^\beta \cdot 1 = b^\beta$ for all $n \in \mathbb{N}$.
    Then, we want to proof $b^{\beta + \frac{1}{n}} < y$, thus we check the inequality $b^{\beta + \frac{1}{n}} < y$ for some $n \in \mathbb{N}$.
    \begin{quote}
        {\bf Proof.} $b^{\beta + \frac{1}{n}} = b^\beta b^{\frac{1}{n}}$. 
        Since $b^{\frac{1}{n}} = (1+c)^\frac{1}{n} \leq 1 + \dfrac{1}{n} c$ and $\dfrac{y}{b^\beta}$ is a positive real number which is greater than $1$, 
        by Archimedeam Property there exists some $n \in \mathbb{N}$ s.t. $n > \dfrac{c\cdot b^\beta}{y-b^\beta} > 0$. In the other world $1 + \dfrac{c}{n} < \dfrac{y}{b^\beta}$.
        Therefore, $b^{\beta + \frac{1}{n}} < y$.
    \end{quote}
    
    Thus, $b^{\beta + \frac{1}{n}}\in A\Rightarrow b^\beta$ is not a upper bound of $A$.
    Therefore, $b^\beta$ should greater or equal to $y$.
    
    If $b^\beta > y$, assume exists $n \in \mathbb{N}$ s.t. $b^{\beta - \frac{1}{n}} > y$.
    Then, $b^{\beta - \frac{1}{n}} = \dfrac{b^\beta}{b^\frac{1}{n}} \geq \dfrac{b^\beta}{1 + \frac{1}{n}c}$. And since $\dfrac{b^\beta}{y}-1$ is a positive real number by $b^\beta > y$, 
    there exists some $n \in \mathbb{n}$ s.t. $\dfrac{c}{n} > \dfrac{b^\beta}{y}-1$ by Archimedeam Property. Therefore, $b^\beta$ is not the least upper bound of $A$(contradiction).

    These two line implies the supremun of $A$, $\beta$, is the unique real number s.t. $b^\beta = y$ since the supremun is unique.
\end{enumerate}
\end{document}
