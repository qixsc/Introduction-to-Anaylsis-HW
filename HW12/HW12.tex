\documentclass[12pt]{article}

\topmargin -40pt
\marginparwidth 0pt
\oddsidemargin  -40pt
\evensidemargin 0pt
\marginparsep 0pt
\linespread{1.9}
\textwidth 7.2 in
 \textheight  10 in
 \hoffset  0.1in

\usepackage{amsthm,amsmath,amssymb,amscd,verbatim,epsfig}
\usepackage{amsopn}
\usepackage{mathptmx}
\usepackage{amsfonts}
%\usepackage{setapace}
\usepackage{graphicx}
\usepackage{bm}
%\usepackage{CJK}
\usepackage{ulem}
\usepackage{multicol}
\usepackage{enumerate}
\usepackage{float}
\usepackage{fontspec}
\usepackage{xeCJK}
\setmainfont{Times New Roman}
\setCJKmainfont{TaipeiSansTCBeta-Regular}
\XeTeXlinebreaklocale "zh"
\XeTeXlinebreakskip = 0pt plus 1pt

\DeclareMathOperator{\closure}{cl}
\DeclareMathOperator{\interior}{int}

\title{Homework 12 of Introduction to Analysis (I), Honor Class}
\author{AM15 黃琦翔 111652028}

\begin{document}
\maketitle
\begin{enumerate}
    \item We claim that $a_n$ is a Cauchy. 
    Then, for $m > n\geq N$
    \begin{align*}
        |f(\dfrac{1}{n}) - f(\dfrac{1}{m})| &< |(\dfrac{1}{n} - \dfrac{1}{m})\cdot f'(c)| \text{ for some } c\in [\dfrac{1}{m}, \dfrac{1}{n}]\\
        &< \dfrac{1}{n} - \dfrac{1}{m}\\
        &< \dfrac{1}{n} \leq \dfrac{1}{N}
    \end{align*}

    Thus, for any $\epsilon > 0$, we take $N > \dfrac{1}{\epsilon}$.
    Therefore, $a_n$ is Cauchy$\implies \displaystyle\lim_{n\to\infty} a_n$ exists.

    \item Since $f'(x)$ exists on $(a, b)$, we can find $s_1 = \sup\{f'(x) \mid x\in (a, b)\},\ s_2 = \displaystyle\lim_{x\to a^+} f(x)$ and $s = \max\{ s_1, s_2\}$.
    
    Then, for $\epsilon > 0$, we take $\delta = \dfrac{\epsilon}{s}$, then $|f(x) - f(x_0)| < |x - x_0| s < \delta s= \epsilon$.
    Thus, $f$ is uniform continuous.

    \item \begin{enumerate}
        \item Claim that for any $\epsilon > 0$, there exists a $N\in \mathbb{N}$ s.t. $|\dfrac{f(x+h) - f(x)}{h} - b| < \epsilon$ for $x > N$ and all $h$.

        Suppose $b \geq 0$.
        Since $f'(x) \to b$ as $x\to \infty$, for any $\epsilon > 0$, exists a $N_0\in \mathbb{N}$ s.t. $|f'(x) - b| < \epsilon$ for all $x > N_0$.
        Thus, $b - \epsilon < f'(x) < b + \epsilon$ for all $x > N_0$.
        By MVT, $f(x + h) = f(x) + h\cdot f'(c)$ for all $h$ and some $c \in [x + h, x]$ or $c\in [x, x+h]$.
        Thus, $|f(x + h) - f(x)| < |h|(b + \epsilon)$ for all $h$ and $x > N_0$.
        Then, we can get the result by taking $N = N_0$.

        If $b < 0$, we can get the same result $|f(x+h) - f(x)| < |h||b-\epsilon|$ by the silmilar way.

        Therefore, $\displaystyle\lim_{x\to\infty} \dfrac{f(x + h) - f(x)}{h} - b = 0$.

        \newpage
        \item Since $f(x) \to a$, for $\epsilon, h> 0$, exsits $N\in\mathbb{N}$ s.t. $|f(x) - a| < \dfrac{\epsilon \cdot h}{2}$.
        
        Then, for $\displaystyle\lim_{x\to\infty} f'(x) = \displaystyle\lim_{x\to\infty} |\dfrac{f(x + h)-f(x)}{h}| \leq \displaystyle\lim_{x\to\infty} \dfrac{|f(x + h) - a| + |a - f(x)|}{h} < 2\dfrac{\epsilon\cdot h}{2h} = \epsilon$.

        Thus, $f'(x) \to 0$ as $x\to \infty$

        \item For any $\epsilon > 0$, there exists a $N\in\mathbb{N}$ s.t. $|f'(x) - b| < \epsilon$ for all $x > N$.
        Then, for any $x_0 > N$
        \begin{align*}
            \lim_{x\to\infty}|\dfrac{f(x)}{x} - b| &= \lim_{x\to\infty}|\dfrac{f(x_0) + f'(x_1)(x-x_0)}{x} - b|\\
            &=\lim_{x\to\infty} |\dfrac{f(x_0)- f'(x_1)x_0}{x}| + |f'(x_1) - b|\text{ for } x_1\in [x_0, x]\\
            &< 0 + \epsilon = \epsilon
        \end{align*}

        Thus, $\dfrac{f(x)}{x} \to b$ as $x\to\infty$.
    \end{enumerate}

    \item\begin{enumerate}
        \item \begin{align*}
            \lim_{h\to 0} \dfrac{f(a+2h)-2f(a+h)+f(a)}{h^2} &= \lim_{h\to 0} \dfrac{f'(a+2h) - f'(a+h)}{h}\\
            &= 2\lim_{h\to 0} \dfrac{f'(a + 2h) - f'(a)}{2h} - \lim_{h\to 0}\dfrac{f'(a + h) - f'(a)}{h}\\
            &= 2f''(a) - f''(a) = f''(a)
        \end{align*}    

        \item \begin{align*}
            \lim_{h\to 0} \dfrac{f(a + 3h) - 3f(a + 2h) + 3f(a + h) - f(a)}{h^3} &= \lim_{h\to 0} \dfrac{3f'(a+3h) - 6f'(a + 2h) + 3f(a + h)}{3h^2}\\
            &= \lim_{h\to 0} \dfrac{f'(a + 3h) - 2f'(a+2h) + f'(a + h)}{h^2}\\
            &= \lim_{h\to 0} \dfrac{3f''(a + 3h) - 4f''(a + 2h) + f''(a+h)}{2h}\\
            &= \lim_{h\to 0} \dfrac{3f''(a+3h) - 3f''(a)}{2h} - \dfrac{4f''(a+2h) - 4f''(a)}{2h}\\
            &\quad+ \dfrac{f''(a + h) - f''(a)}{2h}\\
            &= (\dfrac{9}{2} - 4 + \dfrac{1}{2}) f^{(3)}(a)\\
            &= f^{(3)}(a)
        \end{align*}

    \end{enumerate}
\end{enumerate}
\end{document}