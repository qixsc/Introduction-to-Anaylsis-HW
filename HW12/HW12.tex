\documentclass[12pt]{article}

\topmargin -40pt
\marginparwidth 0pt
\oddsidemargin  -40pt
\evensidemargin 0pt
\marginparsep 0pt
\linespread{1.9}
\textwidth 7.2 in
 \textheight  10 in
 \hoffset  0.1in

\usepackage{amsthm,amsmath,amssymb,amscd,verbatim,epsfig}
\usepackage{amsopn}
\usepackage{mathptmx}
\usepackage{amsfonts}
%\usepackage{setapace}
\usepackage{graphicx}
\usepackage{bm}
%\usepackage{CJK}
\usepackage{ulem}
\usepackage{multicol}
\usepackage{enumerate}
\usepackage{float}
\usepackage{fontspec}
\usepackage{xeCJK}
\setmainfont{Times New Roman}
\setCJKmainfont{TaipeiSansTCBeta-Regular}
\XeTeXlinebreaklocale "zh"
\XeTeXlinebreakskip = 0pt plus 1pt

\DeclareMathOperator{\closure}{cl}
\DeclareMathOperator{\interior}{int}

\title{Homework 12 of Introduction to Analysis (I), Honor Class}
\author{AM15 黃琦翔 111652028}

\begin{document}
\maketitle
\begin{enumerate}
    \item We claim that $a_n$ is a Cauchy. 
    Then, for $m > n\geq N$
    \begin{align*}
        |f(\dfrac{1}{n}) - f(\dfrac{1}{m})| &< |(\dfrac{1}{n} - \dfrac{1}{m})\cdot f'(c)| \text{ for some } c\in [\dfrac{1}{m}, \dfrac{1}{n}]\\
        &< \dfrac{1}{n} - \dfrac{1}{m}\\
        &< \dfrac{1}{n}\\
        &\leq \dfrac{1}{N}
    \end{align*}

    Thus, for any $\epsilon > 0$, we take $N > \dfrac{1}{\epsilon}$.
    Therefore, $a_n$ is Cauchy$\implies \displaystyle\lim_{n\to\infty} a_n$ exists.

    \item Since $f'(x)$ exists on $(a, b)$, we can find $s_1 = \sup\{f'(x) \mid x\in (a, b)\},\ s_2 = \displaystyle\lim_{x\to a^+} f(x)$ and $s = \max\{ s_1, s_2\}$.
    
    Then, for $\epsilon > 0$, we take $\delta = \dfrac{\epsilon}{s}$, then $|f(x) - f(x_0)| < |x - x_0| s < \delta s= \epsilon$.
    Thus, $f$ is uniform continuous.

    \item \begin{enumerate}
        \item We want to show that when $x\to \infty$, $f(x+h)-f(x) \to hb$ for all $h$.
        
        Since $f'(x) \to b$ as $x\to \infty$, for any $\epsilon > 0$, exists a $N\in \mathbb{N}$ s.t. $|f'(x) - b| < \epsilon$ for all $x > N$.
        Thus, for $x > N$, $(b - \epsilon)h < f(x + h) - f(x) < (b + \epsilon)h$. Thus, $|\dfrac{f(x + h) - f(x)}{h} - b| < \epsilon$ for $x > N$.

        Therefore, $\displaystyle\lim_{x\to\infty} \dfrac{f(x + h) - f(x)}{h} - b = 0$.

        \item Since $f(x) \to a$, we assume $f'(x)$
    \end{enumerate}
\end{enumerate}
\end{document}