\documentclass[12pt]{article}

\topmargin -40pt
\marginparwidth 0pt
\oddsidemargin  -40pt
\evensidemargin 0pt
\marginparsep 0pt
\linespread{1.9}
\textwidth 7.2 in
 \textheight  10 in
 \hoffset  0.1in

\usepackage{amsthm,amsmath,amssymb,amscd,verbatim,epsfig}
\usepackage{amsopn}
\usepackage{mathptmx}
\usepackage{amsfonts}
%\usepackage{setapace}
\usepackage{graphicx}
\usepackage{bm}
%\usepackage{CJK}
\usepackage{ulem}
\usepackage{multicol}
\usepackage{enumerate}
\usepackage{float}
\usepackage{fontspec}
\usepackage{xeCJK}
\setmainfont{Times New Roman}
\setCJKmainfont{TaipeiSansTCBeta-Regular}
\XeTeXlinebreaklocale "zh"
\XeTeXlinebreakskip = 0pt plus 1pt

\DeclareMathOperator{\closure}{cl}
\DeclareMathOperator{\interior}{int}

\title{Homework 9 of Introduction to Analysis (I), Honor Class}
\author{AM15 黃琦翔 111652028}

\begin{document}
\maketitle
\begin{enumerate}
    \item Let $\lbrace a_k^+\rbrace$ be the positive elements in $\lbrace a_k\rbrace$, and $\lbrace a_k^-\rbrace$ be the negative ones.
    
    By Riemann's Theorem, if $\displaystyle\sum a_k$ is c.c., for any $i \in \mathbb{N}$,
    we can find smallest $p_i, q_i\in \mathbb{N}$ s.t. $i+1 < \displaystyle\sum_{k=1}^{p_{i+1}} a_k^+ + \displaystyle\sum_{k=1}^{q_i} a_k^-$
    and $i+1 < \displaystyle\sum_{k=1}^{p_{i+1}} a_k^+ + \displaystyle\sum_{k=1}^{q_{i+1}} a_k^-$.

    Then, $i < \displaystyle\sum_{k=1}^{p_i} a_k^+ + \displaystyle\sum_{k=1}^{q_i} a_k^-$ for all $i\in \mathbb{N}$.
    Thus, we can find a rearrangement which partial sum diverges to infinity.

    \item Since $\sum \sqrt{a_n a_{n+1}} \leq \sum \dfrac{a_n + a_{n+1}}{2} = \sum a_n - \dfrac{a_1}{2}$.
    By comparison test, $\sum \sqrt{a_n a_{n+1}}$ converges.

    Since $a_n$ is monotoneic, $a_n$ are all greater than $0$ or all lower than $0$.
    Then, we suppose $a_n$ are greater than $0$ and monotone decreasing.
    $\sum \sqrt{a_n a_{n+1}} \geq \sum a_{n+1} = \sum a_n - a_1$.
    Then, by comparison test, $\sum a_n$ converges.

    \item \begin{enumerate}
        \item \begin{align*}
            \dfrac{a_m}{r_m} + \dfrac{a_{m+1}}{r_{m+1}} + \cdot + \dfrac{a_n}{r_n} &> \dfrac{a_m + a_{m+1} + \cdot a_m}{a_m}\\
            &= \dfrac{r_m - r_n}{r_m}\\
            &= 1 - \dfrac{r_n}{r_m}
        \end{align*}

        \newpage
        Then, we want to show $\sum \dfrac{a_n}{r_n}$ is not Cauchy.
        For any $m \in \mathbb{N}$, we can find a $n \in \mathbb{N}$, $n > m$ s.t. 
        $r_n = \displaystyle\sum_{k=n}^{\infty} a_k > \dfrac{1}{2} \displaystyle\sum_{k=m}^{\infty} a_k = \dfrac{1}{2}r_m$
        since $\displaystyle\sum a_k$ converges.

        Thus, for $N \in \mathbb{N}$ and $m > N$, we can find a $n > m$ s.t. 
        $\displaystyle\sum_{k=m}^{n} \dfrac{a_k}{r_k} > 1- \dfrac{r_n}{r_m} > 1 - \dfrac{1}{2} = \dfrac{1}{2}$.
        Therefore, $\displaystyle\sum \dfrac{a_k}{r_k}$ is not Cauchy implies it diverges.


        \item \begin{align*}
            2(\sqrt{r_n} - \sqrt{r_{n+1}}) &= 2\dfrac{(\sqrt{r_n})(\sqrt{r_n} -\sqrt{r_n + 1})}{\sqrt{r_n}}\\
            &= 2 \dfrac{(\sqrt{r_n})^2 - (\sqrt{r_n}\sqrt{r_{n + 1}})}{\sqrt{r_n}}\\
            &\geq 2\dfrac{r_n - r_{n + 1}}{\sqrt{r_n}}\\
            &= 2\dfrac{a_n}{r_n}\\
            &> \dfrac{a_n}{r_n}
        \end{align*}
        
        We want to proof $\displaystyle\sum_{n=m}^{\infty} \dfrac{a_n}{\sqrt{r_n}} \to \infty$ as $m \to \infty$.
        $\displaystyle\sum_{n=m}^{\infty} \dfrac{a_n}{\sqrt{r_n}} < \displaystyle\sum_{n=m}^{\infty} 2(\sqrt{r_n} - \sqrt{r_{n+1}}) < 2\sqrt{r_m}$.
        And since $\sum a_n$ converges, $r_n \to 0$ as $n \to \infty$.
        Thus, by comparison test, $\sum \dfrac{a_n}{\sqrt{r_n}}$ converges.
    \end{enumerate}

    \item We want to show $\sum a_n - \displaystyle\lim_{n\to 1^-} \sum a_nx^n = \displaystyle\lim_{n\to i^-}\sum  a_n (1 - x^n) = 0$.
    Since $\sum a_n$ converges, $a_n\to 0$ as $n\to \infty$.
    
    Then, let $x = 1-\epsilon$, then $x \to 1- \implies \epsilon \to 0$.
    $\sum a_n(1-x^n) = \sum a_n \cdot \epsilon^n < \epsilon\cdot\sum a_n \to 0$ as $\epsilon \to 0$.
    Thus, $\displaystyle\lim_{x\to 1^-} \sum a_n x^n = \sum a_n$.
\end{enumerate}
\end{document}